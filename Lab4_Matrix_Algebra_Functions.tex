\documentclass[]{article}
\usepackage{lmodern}
\usepackage{amssymb,amsmath}
\usepackage{ifxetex,ifluatex}
\usepackage{fixltx2e} % provides \textsubscript
\ifnum 0\ifxetex 1\fi\ifluatex 1\fi=0 % if pdftex
  \usepackage[T1]{fontenc}
  \usepackage[utf8]{inputenc}
\else % if luatex or xelatex
  \ifxetex
    \usepackage{mathspec}
  \else
    \usepackage{fontspec}
  \fi
  \defaultfontfeatures{Ligatures=TeX,Scale=MatchLowercase}
\fi
% use upquote if available, for straight quotes in verbatim environments
\IfFileExists{upquote.sty}{\usepackage{upquote}}{}
% use microtype if available
\IfFileExists{microtype.sty}{%
\usepackage{microtype}
\UseMicrotypeSet[protrusion]{basicmath} % disable protrusion for tt fonts
}{}
\usepackage[margin=1in]{geometry}
\usepackage{hyperref}
\hypersetup{unicode=true,
            pdftitle={Example Rmarkdown Notebook},
            pdfauthor={Adam Lauretig},
            pdfborder={0 0 0},
            breaklinks=true}
\urlstyle{same}  % don't use monospace font for urls
\usepackage{color}
\usepackage{fancyvrb}
\newcommand{\VerbBar}{|}
\newcommand{\VERB}{\Verb[commandchars=\\\{\}]}
\DefineVerbatimEnvironment{Highlighting}{Verbatim}{commandchars=\\\{\}}
% Add ',fontsize=\small' for more characters per line
\usepackage{framed}
\definecolor{shadecolor}{RGB}{248,248,248}
\newenvironment{Shaded}{\begin{snugshade}}{\end{snugshade}}
\newcommand{\KeywordTok}[1]{\textcolor[rgb]{0.13,0.29,0.53}{\textbf{#1}}}
\newcommand{\DataTypeTok}[1]{\textcolor[rgb]{0.13,0.29,0.53}{#1}}
\newcommand{\DecValTok}[1]{\textcolor[rgb]{0.00,0.00,0.81}{#1}}
\newcommand{\BaseNTok}[1]{\textcolor[rgb]{0.00,0.00,0.81}{#1}}
\newcommand{\FloatTok}[1]{\textcolor[rgb]{0.00,0.00,0.81}{#1}}
\newcommand{\ConstantTok}[1]{\textcolor[rgb]{0.00,0.00,0.00}{#1}}
\newcommand{\CharTok}[1]{\textcolor[rgb]{0.31,0.60,0.02}{#1}}
\newcommand{\SpecialCharTok}[1]{\textcolor[rgb]{0.00,0.00,0.00}{#1}}
\newcommand{\StringTok}[1]{\textcolor[rgb]{0.31,0.60,0.02}{#1}}
\newcommand{\VerbatimStringTok}[1]{\textcolor[rgb]{0.31,0.60,0.02}{#1}}
\newcommand{\SpecialStringTok}[1]{\textcolor[rgb]{0.31,0.60,0.02}{#1}}
\newcommand{\ImportTok}[1]{#1}
\newcommand{\CommentTok}[1]{\textcolor[rgb]{0.56,0.35,0.01}{\textit{#1}}}
\newcommand{\DocumentationTok}[1]{\textcolor[rgb]{0.56,0.35,0.01}{\textbf{\textit{#1}}}}
\newcommand{\AnnotationTok}[1]{\textcolor[rgb]{0.56,0.35,0.01}{\textbf{\textit{#1}}}}
\newcommand{\CommentVarTok}[1]{\textcolor[rgb]{0.56,0.35,0.01}{\textbf{\textit{#1}}}}
\newcommand{\OtherTok}[1]{\textcolor[rgb]{0.56,0.35,0.01}{#1}}
\newcommand{\FunctionTok}[1]{\textcolor[rgb]{0.00,0.00,0.00}{#1}}
\newcommand{\VariableTok}[1]{\textcolor[rgb]{0.00,0.00,0.00}{#1}}
\newcommand{\ControlFlowTok}[1]{\textcolor[rgb]{0.13,0.29,0.53}{\textbf{#1}}}
\newcommand{\OperatorTok}[1]{\textcolor[rgb]{0.81,0.36,0.00}{\textbf{#1}}}
\newcommand{\BuiltInTok}[1]{#1}
\newcommand{\ExtensionTok}[1]{#1}
\newcommand{\PreprocessorTok}[1]{\textcolor[rgb]{0.56,0.35,0.01}{\textit{#1}}}
\newcommand{\AttributeTok}[1]{\textcolor[rgb]{0.77,0.63,0.00}{#1}}
\newcommand{\RegionMarkerTok}[1]{#1}
\newcommand{\InformationTok}[1]{\textcolor[rgb]{0.56,0.35,0.01}{\textbf{\textit{#1}}}}
\newcommand{\WarningTok}[1]{\textcolor[rgb]{0.56,0.35,0.01}{\textbf{\textit{#1}}}}
\newcommand{\AlertTok}[1]{\textcolor[rgb]{0.94,0.16,0.16}{#1}}
\newcommand{\ErrorTok}[1]{\textcolor[rgb]{0.64,0.00,0.00}{\textbf{#1}}}
\newcommand{\NormalTok}[1]{#1}
\usepackage{graphicx,grffile}
\makeatletter
\def\maxwidth{\ifdim\Gin@nat@width>\linewidth\linewidth\else\Gin@nat@width\fi}
\def\maxheight{\ifdim\Gin@nat@height>\textheight\textheight\else\Gin@nat@height\fi}
\makeatother
% Scale images if necessary, so that they will not overflow the page
% margins by default, and it is still possible to overwrite the defaults
% using explicit options in \includegraphics[width, height, ...]{}
\setkeys{Gin}{width=\maxwidth,height=\maxheight,keepaspectratio}
\IfFileExists{parskip.sty}{%
\usepackage{parskip}
}{% else
\setlength{\parindent}{0pt}
\setlength{\parskip}{6pt plus 2pt minus 1pt}
}
\setlength{\emergencystretch}{3em}  % prevent overfull lines
\providecommand{\tightlist}{%
  \setlength{\itemsep}{0pt}\setlength{\parskip}{0pt}}
\setcounter{secnumdepth}{0}
% Redefines (sub)paragraphs to behave more like sections
\ifx\paragraph\undefined\else
\let\oldparagraph\paragraph
\renewcommand{\paragraph}[1]{\oldparagraph{#1}\mbox{}}
\fi
\ifx\subparagraph\undefined\else
\let\oldsubparagraph\subparagraph
\renewcommand{\subparagraph}[1]{\oldsubparagraph{#1}\mbox{}}
\fi

%%% Use protect on footnotes to avoid problems with footnotes in titles
\let\rmarkdownfootnote\footnote%
\def\footnote{\protect\rmarkdownfootnote}

%%% Change title format to be more compact
\usepackage{titling}

% Create subtitle command for use in maketitle
\newcommand{\subtitle}[1]{
  \posttitle{
    \begin{center}\large#1\end{center}
    }
}

\setlength{\droptitle}{-2em}
  \title{Example Rmarkdown Notebook}
  \pretitle{\vspace{\droptitle}\centering\huge}
  \posttitle{\par}
  \author{Adam Lauretig}
  \preauthor{\centering\large\emph}
  \postauthor{\par}
  \predate{\centering\large\emph}
  \postdate{\par}
  \date{\today}

\usepackage{palatino}
\usepackage{graphicx}
\usepackage{scrextend}

\begin{document}
\maketitle

\section{Matrix Algebra and
Functions}\label{matrix-algebra-and-functions}

There are five basic data structures in R: vectors, matrices, arrays,
lists, and data.frames. We'll be going through each of these here, but
if you want an in depth exploration of these I'd recommend Norman
Matloff's \emph{The Art of R Programming: A Tour of Statistical Software
Design}.

\subsection{Matrix basics}\label{matrix-basics}

Up to this point, we've primarily \emph{talked} about vectors. We've
encountered other data types, but haven't used them. Vectors have
length, but no width (they can only represent one variable at a time).
Matrices are just collections of vectors (exactly like you learned in
math camp). We can combine them by column using \texttt{cbind}, or by
row, using \texttt{rbind}. We then access elements of matrix by
\texttt{matrix[row, column]}.

\begin{Shaded}
\begin{Highlighting}[]
\NormalTok{vap <-}\StringTok{ }\NormalTok{voting.age.population <-}\StringTok{ }\KeywordTok{c}\NormalTok{(}\DecValTok{3481823}\NormalTok{, }\DecValTok{496387}\NormalTok{, }\DecValTok{4582842}\NormalTok{, }\DecValTok{2120139}\NormalTok{,}\DecValTok{26955438}\NormalTok{,}\DecValTok{3617942}\NormalTok{,}\DecValTok{2673154}\NormalTok{,}\DecValTok{652189}\NormalTok{,}\DecValTok{472143}\NormalTok{,}\DecValTok{14085749}\NormalTok{,}\DecValTok{6915512}\NormalTok{,}\DecValTok{995937}\NormalTok{,}\DecValTok{1073799}\NormalTok{,}\DecValTok{9600372}\NormalTok{,}\DecValTok{4732010}\NormalTok{,}\DecValTok{2265860}\NormalTok{,}\DecValTok{2068253}\NormalTok{,}\DecValTok{3213141}\NormalTok{,}\DecValTok{3188765}\NormalTok{,}\DecValTok{1033632}\NormalTok{,}\DecValTok{4242214}\NormalTok{,}\DecValTok{4997677}\NormalTok{,}\DecValTok{7620982}\NormalTok{,}\DecValTok{3908159}\NormalTok{,}\DecValTok{2139918}\NormalTok{,}\DecValTok{4426278}\NormalTok{,}\DecValTok{731365}\NormalTok{,}\DecValTok{1321923}\NormalTok{,}\DecValTok{1870315}\NormalTok{,}\DecValTok{1012033}\NormalTok{,}\DecValTok{6598368}\NormalTok{,}\DecValTok{1452962}\NormalTok{,}\DecValTok{14838076}\NormalTok{,}\DecValTok{6752018}\NormalTok{,}\DecValTok{494923}\NormalTok{,}\DecValTok{8697456}\NormalTok{,}\DecValTok{2697855}\NormalTok{,}\DecValTok{2850525}\NormalTok{,}\DecValTok{9612380}\NormalTok{,}\DecValTok{824854}\NormalTok{,}\DecValTok{3303593}\NormalTok{,}\DecValTok{594599}\NormalTok{,}\DecValTok{4636679}\NormalTok{,}\DecValTok{17038979}\NormalTok{,}\DecValTok{1797941}\NormalTok{,}\DecValTok{487900}\NormalTok{,}\DecValTok{5841335}\NormalTok{,}\DecValTok{4876661}\NormalTok{,}\DecValTok{1421717}\NormalTok{,}\DecValTok{4257230}\NormalTok{,}\DecValTok{392344}\NormalTok{)}

\NormalTok{total.votes <-}\StringTok{ }\NormalTok{tv <-}\StringTok{ }\KeywordTok{c}\NormalTok{(}\OtherTok{NA}\NormalTok{, }\DecValTok{238307}\NormalTok{, }\DecValTok{1553032}\NormalTok{, }\DecValTok{780409}\NormalTok{,}\DecValTok{8899059}\NormalTok{,}\DecValTok{1586105}\NormalTok{, }\DecValTok{1162391}\NormalTok{,}\DecValTok{258053}\NormalTok{, }\DecValTok{122356}\NormalTok{,}\DecValTok{4884544}\NormalTok{, }\DecValTok{2143845}\NormalTok{,}\DecValTok{348988}\NormalTok{, }\DecValTok{458927}\NormalTok{,}\DecValTok{3586292}\NormalTok{, }\DecValTok{1719351}\NormalTok{,}\DecValTok{1071509}\NormalTok{, }\DecValTok{864083}\NormalTok{,}\DecValTok{1370062}\NormalTok{, }\DecValTok{954896}\NormalTok{,}\OtherTok{NA}\NormalTok{, }\DecValTok{1809237}\NormalTok{, }\DecValTok{2243835}\NormalTok{,}\DecValTok{3852008}\NormalTok{, }\DecValTok{2217552}\NormalTok{,}\OtherTok{NA}\NormalTok{, }\DecValTok{2178278}\NormalTok{, }\DecValTok{411061}\NormalTok{,}\DecValTok{610499}\NormalTok{, }\DecValTok{586274}\NormalTok{,}\DecValTok{418550}\NormalTok{, }\DecValTok{2315643}\NormalTok{,}\DecValTok{568597}\NormalTok{, }\DecValTok{4703830}\NormalTok{,}\DecValTok{2036451}\NormalTok{, }\DecValTok{220479}\NormalTok{,}\DecValTok{4184072}\NormalTok{, }\OtherTok{NA}\NormalTok{,}\DecValTok{1399650}\NormalTok{, }\OtherTok{NA}\NormalTok{,}\DecValTok{392882}\NormalTok{, }\DecValTok{1117311}\NormalTok{,}\DecValTok{341105}\NormalTok{, }\DecValTok{1868363}\NormalTok{,}\OtherTok{NA}\NormalTok{, }\DecValTok{582561}\NormalTok{, }\DecValTok{263025}\NormalTok{,}\DecValTok{2398589}\NormalTok{, }\DecValTok{2085074}\NormalTok{,}\DecValTok{473014}\NormalTok{, }\DecValTok{2183155}\NormalTok{, }\DecValTok{196217}\NormalTok{)}

\NormalTok{m1 <-}\StringTok{ }\KeywordTok{cbind}\NormalTok{(vap, tv) }\CommentTok{# Combined by column}
\NormalTok{m2 <-}\StringTok{ }\KeywordTok{rbind}\NormalTok{(vap, tv) }\CommentTok{# combined by row}
\NormalTok{m2[}\DecValTok{1}\NormalTok{,}\DecValTok{2}\NormalTok{] }\CommentTok{# first row, second column}
\end{Highlighting}
\end{Shaded}

\begin{verbatim}
##    vap 
## 496387
\end{verbatim}

\begin{Shaded}
\begin{Highlighting}[]
\NormalTok{m1[,}\DecValTok{1}\NormalTok{] }\CommentTok{# the ith colum}
\end{Highlighting}
\end{Shaded}

\begin{verbatim}
##  [1]  3481823   496387  4582842  2120139 26955438  3617942  2673154
##  [8]   652189   472143 14085749  6915512   995937  1073799  9600372
## [15]  4732010  2265860  2068253  3213141  3188765  1033632  4242214
## [22]  4997677  7620982  3908159  2139918  4426278   731365  1321923
## [29]  1870315  1012033  6598368  1452962 14838076  6752018   494923
## [36]  8697456  2697855  2850525  9612380   824854  3303593   594599
## [43]  4636679 17038979  1797941   487900  5841335  4876661  1421717
## [50]  4257230   392344
\end{verbatim}

\begin{Shaded}
\begin{Highlighting}[]
\NormalTok{m1[}\DecValTok{1}\OperatorTok{:}\DecValTok{5}\NormalTok{,}\DecValTok{1}\OperatorTok{:}\DecValTok{2}\NormalTok{] }\CommentTok{# a submatrix}
\end{Highlighting}
\end{Shaded}

\begin{verbatim}
##           vap      tv
## [1,]  3481823      NA
## [2,]   496387  238307
## [3,]  4582842 1553032
## [4,]  2120139  780409
## [5,] 26955438 8899059
\end{verbatim}

\begin{Shaded}
\begin{Highlighting}[]
\NormalTok{m2[}\DecValTok{1}\NormalTok{,}\DecValTok{1}\OperatorTok{:}\DecValTok{10}\NormalTok{]}
\end{Highlighting}
\end{Shaded}

\begin{verbatim}
##  [1]  3481823   496387  4582842  2120139 26955438  3617942  2673154
##  [8]   652189   472143 14085749
\end{verbatim}

\begin{Shaded}
\begin{Highlighting}[]
\NormalTok{m2[}\DecValTok{1}\OperatorTok{:}\DecValTok{2}\NormalTok{, }\DecValTok{1}\OperatorTok{:}\DecValTok{10}\NormalTok{]}
\end{Highlighting}
\end{Shaded}

\begin{verbatim}
##        [,1]   [,2]    [,3]    [,4]     [,5]    [,6]    [,7]   [,8]   [,9]
## vap 3481823 496387 4582842 2120139 26955438 3617942 2673154 652189 472143
## tv       NA 238307 1553032  780409  8899059 1586105 1162391 258053 122356
##        [,10]
## vap 14085749
## tv   4884544
\end{verbatim}

\begin{Shaded}
\begin{Highlighting}[]
\NormalTok{m2[, }\DecValTok{1}\OperatorTok{:}\DecValTok{10}\NormalTok{] }\CommentTok{# same as previous line since there are only two rows.}
\end{Highlighting}
\end{Shaded}

\begin{verbatim}
##        [,1]   [,2]    [,3]    [,4]     [,5]    [,6]    [,7]   [,8]   [,9]
## vap 3481823 496387 4582842 2120139 26955438 3617942 2673154 652189 472143
## tv       NA 238307 1553032  780409  8899059 1586105 1162391 258053 122356
##        [,10]
## vap 14085749
## tv   4884544
\end{verbatim}

\begin{Shaded}
\begin{Highlighting}[]
\KeywordTok{class}\NormalTok{(m2)}
\end{Highlighting}
\end{Shaded}

\begin{verbatim}
## [1] "matrix"
\end{verbatim}

However, we can also create matrices directly, we don't need to create
vectors first:

\begin{Shaded}
\begin{Highlighting}[]
\CommentTok{#Another way to specify a matrix}
\KeywordTok{matrix}\NormalTok{(}\DecValTok{1}\OperatorTok{:}\DecValTok{10}\NormalTok{, }\DataTypeTok{nrow =} \DecValTok{5}\NormalTok{)}
\end{Highlighting}
\end{Shaded}

\begin{verbatim}
##      [,1] [,2]
## [1,]    1    6
## [2,]    2    7
## [3,]    3    8
## [4,]    4    9
## [5,]    5   10
\end{verbatim}

\begin{Shaded}
\begin{Highlighting}[]
\KeywordTok{matrix}\NormalTok{(}\DecValTok{1}\OperatorTok{:}\DecValTok{10}\NormalTok{, }\DataTypeTok{ncol =} \DecValTok{2}\NormalTok{) }\CommentTok{#the same}
\end{Highlighting}
\end{Shaded}

\begin{verbatim}
##      [,1] [,2]
## [1,]    1    6
## [2,]    2    7
## [3,]    3    8
## [4,]    4    9
## [5,]    5   10
\end{verbatim}

\begin{Shaded}
\begin{Highlighting}[]
\KeywordTok{matrix}\NormalTok{(}\DecValTok{1}\OperatorTok{:}\DecValTok{10}\NormalTok{, }\DataTypeTok{nrow =} \DecValTok{5}\NormalTok{, }\DataTypeTok{ncol =} \DecValTok{2}\NormalTok{) }\CommentTok{# the same}
\end{Highlighting}
\end{Shaded}

\begin{verbatim}
##      [,1] [,2]
## [1,]    1    6
## [2,]    2    7
## [3,]    3    8
## [4,]    4    9
## [5,]    5   10
\end{verbatim}

\begin{Shaded}
\begin{Highlighting}[]
\KeywordTok{matrix}\NormalTok{(}\DecValTok{1}\OperatorTok{:}\DecValTok{10}\NormalTok{, }\DataTypeTok{nrow =} \DecValTok{5}\NormalTok{, }\DataTypeTok{byrow =} \OtherTok{TRUE}\NormalTok{) ## not the same}
\end{Highlighting}
\end{Shaded}

\begin{verbatim}
##      [,1] [,2]
## [1,]    1    2
## [2,]    3    4
## [3,]    5    6
## [4,]    7    8
## [5,]    9   10
\end{verbatim}

By default, R will fill each column of a matrix, and then move to the
next one. If you specify \texttt{byrow = TRUE}, however, R will fill
each row, and then move onto the next one.

\subsection{Arrays and attributes}\label{arrays-and-attributes}

Arrays are a more general way to store data. Where a matrix can only
have 2 dimensions (rows and columns), arrays can have an arbitrary
number of dimensions, but this \emph{will} increase the amount of memory
they consume.

Let's examine a cube of dimensions \(3 \times 4 \times 2\). One way of
thinking of this is two \(3 \times 4\) matrices stacked on top of each
other:

\begin{Shaded}
\begin{Highlighting}[]
\NormalTok{a <-}\StringTok{ }\KeywordTok{array}\NormalTok{(}\DecValTok{1}\OperatorTok{:}\DecValTok{24}\NormalTok{, }\DataTypeTok{dim =} \KeywordTok{c}\NormalTok{(}\DecValTok{3}\NormalTok{, }\DecValTok{4}\NormalTok{, }\DecValTok{2}\NormalTok{))}
\NormalTok{a}
\end{Highlighting}
\end{Shaded}

\begin{verbatim}
## , , 1
## 
##      [,1] [,2] [,3] [,4]
## [1,]    1    4    7   10
## [2,]    2    5    8   11
## [3,]    3    6    9   12
## 
## , , 2
## 
##      [,1] [,2] [,3] [,4]
## [1,]   13   16   19   22
## [2,]   14   17   20   23
## [3,]   15   18   21   24
\end{verbatim}

Since this array has three dimensions, there are now three indices we
can use to access the array:

\begin{Shaded}
\begin{Highlighting}[]
\NormalTok{a[, , }\DecValTok{1}\NormalTok{]}
\end{Highlighting}
\end{Shaded}

\begin{verbatim}
##      [,1] [,2] [,3] [,4]
## [1,]    1    4    7   10
## [2,]    2    5    8   11
## [3,]    3    6    9   12
\end{verbatim}

\begin{Shaded}
\begin{Highlighting}[]
\NormalTok{a[, }\DecValTok{1}\NormalTok{, ]}
\end{Highlighting}
\end{Shaded}

\begin{verbatim}
##      [,1] [,2]
## [1,]    1   13
## [2,]    2   14
## [3,]    3   15
\end{verbatim}

\begin{Shaded}
\begin{Highlighting}[]
\NormalTok{a[}\DecValTok{1}\NormalTok{, , ]}
\end{Highlighting}
\end{Shaded}

\begin{verbatim}
##      [,1] [,2]
## [1,]    1   13
## [2,]    4   16
## [3,]    7   19
## [4,]   10   22
\end{verbatim}

\begin{Shaded}
\begin{Highlighting}[]
\NormalTok{a[}\DecValTok{1}\NormalTok{, }\DecValTok{1}\NormalTok{, ]}
\end{Highlighting}
\end{Shaded}

\begin{verbatim}
## [1]  1 13
\end{verbatim}

\begin{Shaded}
\begin{Highlighting}[]
\NormalTok{a[, }\DecValTok{1}\NormalTok{, }\DecValTok{1}\NormalTok{]}
\end{Highlighting}
\end{Shaded}

\begin{verbatim}
## [1] 1 2 3
\end{verbatim}

\begin{Shaded}
\begin{Highlighting}[]
\NormalTok{a[}\DecValTok{1}\NormalTok{, }\DecValTok{1}\NormalTok{, }\DecValTok{1}\NormalTok{]}
\end{Highlighting}
\end{Shaded}

\begin{verbatim}
## [1] 1
\end{verbatim}

Notice that the `dim' is asssigned. This is an ``attribute'' of the
array; attributes are some piece of data associated with the structure
that isn't the data itself, and are used to make working with these data
easier.

\begin{Shaded}
\begin{Highlighting}[]
\KeywordTok{dim}\NormalTok{(a) }
\end{Highlighting}
\end{Shaded}

\begin{verbatim}
## [1] 3 4 2
\end{verbatim}

\begin{Shaded}
\begin{Highlighting}[]
\KeywordTok{attributes}\NormalTok{(a)}
\end{Highlighting}
\end{Shaded}

\begin{verbatim}
## $dim
## [1] 3 4 2
\end{verbatim}

\begin{Shaded}
\begin{Highlighting}[]
\KeywordTok{str}\NormalTok{(a)}
\end{Highlighting}
\end{Shaded}

\begin{verbatim}
##  int [1:3, 1:4, 1:2] 1 2 3 4 5 6 7 8 9 10 ...
\end{verbatim}

Matrices also have this attribute (\texttt{dim}), and also have and
attribute \texttt{dimnames()}, which are strings (technically lists of
strings, but we'll get to that in a minute), which allow you to label
your observations.

\begin{Shaded}
\begin{Highlighting}[]
\KeywordTok{dim}\NormalTok{(m1) }\CommentTok{# number of rows, number of columns}
\end{Highlighting}
\end{Shaded}

\begin{verbatim}
## [1] 51  2
\end{verbatim}

\begin{Shaded}
\begin{Highlighting}[]
\KeywordTok{attributes}\NormalTok{(m1) }\CommentTok{# there is another attribute here -- the columns have names}
\end{Highlighting}
\end{Shaded}

\begin{verbatim}
## $dim
## [1] 51  2
## 
## $dimnames
## $dimnames[[1]]
## NULL
## 
## $dimnames[[2]]
## [1] "vap" "tv"
\end{verbatim}

\begin{Shaded}
\begin{Highlighting}[]
\KeywordTok{dimnames}\NormalTok{(m1) }\CommentTok{# we can either assign or get the dimnames attribute}
\end{Highlighting}
\end{Shaded}

\begin{verbatim}
## [[1]]
## NULL
## 
## [[2]]
## [1] "vap" "tv"
\end{verbatim}

\begin{Shaded}
\begin{Highlighting}[]
\CommentTok{# The first part is the rownames (which we didn't assign)}
\KeywordTok{dimnames}\NormalTok{(m2) }\CommentTok{# here the columns have no names}
\end{Highlighting}
\end{Shaded}

\begin{verbatim}
## [[1]]
## [1] "vap" "tv" 
## 
## [[2]]
## NULL
\end{verbatim}

\begin{Shaded}
\begin{Highlighting}[]
\KeywordTok{dimnames}\NormalTok{(m1)[[}\DecValTok{2}\NormalTok{]][}\DecValTok{1}\NormalTok{]<-}\StringTok{"Dracula"}
\KeywordTok{head}\NormalTok{(m1) }\CommentTok{# We have re-named the first column to have the name "Dracula"}
\end{Highlighting}
\end{Shaded}

\begin{verbatim}
##       Dracula      tv
## [1,]  3481823      NA
## [2,]   496387  238307
## [3,]  4582842 1553032
## [4,]  2120139  780409
## [5,] 26955438 8899059
## [6,]  3617942 1586105
\end{verbatim}

\begin{Shaded}
\begin{Highlighting}[]
\KeywordTok{dimnames}\NormalTok{(m1)[[}\DecValTok{2}\NormalTok{]][}\DecValTok{1}\NormalTok{]<-}\StringTok{"vap"} \CommentTok{# all of this bracketing is because this is a list ... what's a list?}
\KeywordTok{head}\NormalTok{(m1)}
\end{Highlighting}
\end{Shaded}

\begin{verbatim}
##           vap      tv
## [1,]  3481823      NA
## [2,]   496387  238307
## [3,]  4582842 1553032
## [4,]  2120139  780409
## [5,] 26955438 8899059
## [6,]  3617942 1586105
\end{verbatim}

R is flexible, and there are multiple ways to access dimnames:

\begin{Shaded}
\begin{Highlighting}[]
\CommentTok{# Another way to do this}
\KeywordTok{colnames}\NormalTok{(m1)}
\end{Highlighting}
\end{Shaded}

\begin{verbatim}
## [1] "vap" "tv"
\end{verbatim}

\begin{Shaded}
\begin{Highlighting}[]
\CommentTok{# How would we rename the first column?}
\KeywordTok{colnames}\NormalTok{(m2)}
\end{Highlighting}
\end{Shaded}

\begin{verbatim}
## NULL
\end{verbatim}

\begin{Shaded}
\begin{Highlighting}[]
\KeywordTok{rownames}\NormalTok{(m1)}
\end{Highlighting}
\end{Shaded}

\begin{verbatim}
## NULL
\end{verbatim}

\begin{Shaded}
\begin{Highlighting}[]
\KeywordTok{rownames}\NormalTok{(m2)}
\end{Highlighting}
\end{Shaded}

\begin{verbatim}
## [1] "vap" "tv"
\end{verbatim}

\subsection{Lists}\label{lists}

One downside to matrices and vectors is that every element in them must
be the same type (all numerics, or all intergers, or all character
vectors). Lists offer a way around this restriction, they can combine
multiple data types. Lists are a very flexible way to store data, and
are maybe the most common data structure you'll encounter: many
functions produce lists.

\begin{Shaded}
\begin{Highlighting}[]
\NormalTok{list.a <-}\StringTok{ }\KeywordTok{list}\NormalTok{(m1, vap, }\DecValTok{3}\NormalTok{) }\CommentTok{# m1 is a matrix, vap is a vector, 3 is an integer}
\NormalTok{list.a}
\end{Highlighting}
\end{Shaded}

\begin{verbatim}
## [[1]]
##            vap      tv
##  [1,]  3481823      NA
##  [2,]   496387  238307
##  [3,]  4582842 1553032
##  [4,]  2120139  780409
##  [5,] 26955438 8899059
##  [6,]  3617942 1586105
##  [7,]  2673154 1162391
##  [8,]   652189  258053
##  [9,]   472143  122356
## [10,] 14085749 4884544
## [11,]  6915512 2143845
## [12,]   995937  348988
## [13,]  1073799  458927
## [14,]  9600372 3586292
## [15,]  4732010 1719351
## [16,]  2265860 1071509
## [17,]  2068253  864083
## [18,]  3213141 1370062
## [19,]  3188765  954896
## [20,]  1033632      NA
## [21,]  4242214 1809237
## [22,]  4997677 2243835
## [23,]  7620982 3852008
## [24,]  3908159 2217552
## [25,]  2139918      NA
## [26,]  4426278 2178278
## [27,]   731365  411061
## [28,]  1321923  610499
## [29,]  1870315  586274
## [30,]  1012033  418550
## [31,]  6598368 2315643
## [32,]  1452962  568597
## [33,] 14838076 4703830
## [34,]  6752018 2036451
## [35,]   494923  220479
## [36,]  8697456 4184072
## [37,]  2697855      NA
## [38,]  2850525 1399650
## [39,]  9612380      NA
## [40,]   824854  392882
## [41,]  3303593 1117311
## [42,]   594599  341105
## [43,]  4636679 1868363
## [44,] 17038979      NA
## [45,]  1797941  582561
## [46,]   487900  263025
## [47,]  5841335 2398589
## [48,]  4876661 2085074
## [49,]  1421717  473014
## [50,]  4257230 2183155
## [51,]   392344  196217
## 
## [[2]]
##  [1]  3481823   496387  4582842  2120139 26955438  3617942  2673154
##  [8]   652189   472143 14085749  6915512   995937  1073799  9600372
## [15]  4732010  2265860  2068253  3213141  3188765  1033632  4242214
## [22]  4997677  7620982  3908159  2139918  4426278   731365  1321923
## [29]  1870315  1012033  6598368  1452962 14838076  6752018   494923
## [36]  8697456  2697855  2850525  9612380   824854  3303593   594599
## [43]  4636679 17038979  1797941   487900  5841335  4876661  1421717
## [50]  4257230   392344
## 
## [[3]]
## [1] 3
\end{verbatim}

We can make all sorts of lists, and can even create lists containing
other lists!

\begin{Shaded}
\begin{Highlighting}[]
\NormalTok{vector1 <-}\StringTok{ }\KeywordTok{c}\NormalTok{(}\DecValTok{1}\NormalTok{,}\DecValTok{2}\NormalTok{,}\DecValTok{3}\NormalTok{)}
\NormalTok{gospels <-}\StringTok{ }\KeywordTok{c}\NormalTok{(}\StringTok{"matthew"}\NormalTok{,}\StringTok{"mark"}\NormalTok{,}\StringTok{"luke"}\NormalTok{, }\StringTok{"john"}\NormalTok{)}
\NormalTok{my.matrix <-}\StringTok{ }\KeywordTok{matrix}\NormalTok{(}\KeywordTok{c}\NormalTok{(}\DecValTok{1}\OperatorTok{:}\DecValTok{20}\NormalTok{), }\DataTypeTok{nrow=}\DecValTok{4}\NormalTok{)}
\NormalTok{my.data <-}\StringTok{ }\KeywordTok{data.frame}\NormalTok{(}\KeywordTok{cbind}\NormalTok{(vap, tv))}
\NormalTok{my.crazy.list <-}\StringTok{ }\KeywordTok{list}\NormalTok{(vector1, gospels, my.matrix, }\OtherTok{TRUE}\NormalTok{, list.a)}
\NormalTok{my.crazy.list }\CommentTok{# we can combine anything we want -- we can even include other lists in our lists}
\end{Highlighting}
\end{Shaded}

\begin{verbatim}
## [[1]]
## [1] 1 2 3
## 
## [[2]]
## [1] "matthew" "mark"    "luke"    "john"   
## 
## [[3]]
##      [,1] [,2] [,3] [,4] [,5]
## [1,]    1    5    9   13   17
## [2,]    2    6   10   14   18
## [3,]    3    7   11   15   19
## [4,]    4    8   12   16   20
## 
## [[4]]
## [1] TRUE
## 
## [[5]]
## [[5]][[1]]
##            vap      tv
##  [1,]  3481823      NA
##  [2,]   496387  238307
##  [3,]  4582842 1553032
##  [4,]  2120139  780409
##  [5,] 26955438 8899059
##  [6,]  3617942 1586105
##  [7,]  2673154 1162391
##  [8,]   652189  258053
##  [9,]   472143  122356
## [10,] 14085749 4884544
## [11,]  6915512 2143845
## [12,]   995937  348988
## [13,]  1073799  458927
## [14,]  9600372 3586292
## [15,]  4732010 1719351
## [16,]  2265860 1071509
## [17,]  2068253  864083
## [18,]  3213141 1370062
## [19,]  3188765  954896
## [20,]  1033632      NA
## [21,]  4242214 1809237
## [22,]  4997677 2243835
## [23,]  7620982 3852008
## [24,]  3908159 2217552
## [25,]  2139918      NA
## [26,]  4426278 2178278
## [27,]   731365  411061
## [28,]  1321923  610499
## [29,]  1870315  586274
## [30,]  1012033  418550
## [31,]  6598368 2315643
## [32,]  1452962  568597
## [33,] 14838076 4703830
## [34,]  6752018 2036451
## [35,]   494923  220479
## [36,]  8697456 4184072
## [37,]  2697855      NA
## [38,]  2850525 1399650
## [39,]  9612380      NA
## [40,]   824854  392882
## [41,]  3303593 1117311
## [42,]   594599  341105
## [43,]  4636679 1868363
## [44,] 17038979      NA
## [45,]  1797941  582561
## [46,]   487900  263025
## [47,]  5841335 2398589
## [48,]  4876661 2085074
## [49,]  1421717  473014
## [50,]  4257230 2183155
## [51,]   392344  196217
## 
## [[5]][[2]]
##  [1]  3481823   496387  4582842  2120139 26955438  3617942  2673154
##  [8]   652189   472143 14085749  6915512   995937  1073799  9600372
## [15]  4732010  2265860  2068253  3213141  3188765  1033632  4242214
## [22]  4997677  7620982  3908159  2139918  4426278   731365  1321923
## [29]  1870315  1012033  6598368  1452962 14838076  6752018   494923
## [36]  8697456  2697855  2850525  9612380   824854  3303593   594599
## [43]  4636679 17038979  1797941   487900  5841335  4876661  1421717
## [50]  4257230   392344
## 
## [[5]][[3]]
## [1] 3
\end{verbatim}

What if we want to access the attributes of our list?

\begin{Shaded}
\begin{Highlighting}[]
\KeywordTok{str}\NormalTok{(my.crazy.list) }\CommentTok{# the str() function is useful for looking at the basic components }
\end{Highlighting}
\end{Shaded}

\begin{verbatim}
## List of 5
##  $ : num [1:3] 1 2 3
##  $ : chr [1:4] "matthew" "mark" "luke" "john"
##  $ : int [1:4, 1:5] 1 2 3 4 5 6 7 8 9 10 ...
##  $ : logi TRUE
##  $ :List of 3
##   ..$ : num [1:51, 1:2] 3481823 496387 4582842 2120139 26955438 ...
##   .. ..- attr(*, "dimnames")=List of 2
##   .. .. ..$ : NULL
##   .. .. ..$ : chr [1:2] "vap" "tv"
##   ..$ : num [1:51] 3481823 496387 4582842 2120139 26955438 ...
##   ..$ : num 3
\end{verbatim}

\begin{Shaded}
\begin{Highlighting}[]
\CommentTok{# of any complicated object like this}
\CommentTok{#str() will work with most types of objects}

\KeywordTok{attributes}\NormalTok{(my.crazy.list) }\CommentTok{# lists has attributes, but we haven't set them}
\end{Highlighting}
\end{Shaded}

\begin{verbatim}
## NULL
\end{verbatim}

\begin{Shaded}
\begin{Highlighting}[]
\KeywordTok{length}\NormalTok{(my.crazy.list) }\CommentTok{# this reports the number of major sub-elements in the list}
\end{Highlighting}
\end{Shaded}

\begin{verbatim}
## [1] 5
\end{verbatim}

\begin{Shaded}
\begin{Highlighting}[]
\KeywordTok{dim}\NormalTok{(my.crazy.list) }\CommentTok{# this won't work for complicated lists}
\end{Highlighting}
\end{Shaded}

\begin{verbatim}
## NULL
\end{verbatim}

\begin{Shaded}
\begin{Highlighting}[]
\KeywordTok{names}\NormalTok{(my.crazy.list) <-}\StringTok{ }\KeywordTok{c}\NormalTok{(}\StringTok{"one"}\NormalTok{, }\StringTok{"two"}\NormalTok{, }\StringTok{"three"}\NormalTok{, }\StringTok{"four"}\NormalTok{, }\StringTok{"five"}\NormalTok{)}
\KeywordTok{str}\NormalTok{(my.crazy.list) }\CommentTok{# now each part of the list has a name attribute}
\end{Highlighting}
\end{Shaded}

\begin{verbatim}
## List of 5
##  $ one  : num [1:3] 1 2 3
##  $ two  : chr [1:4] "matthew" "mark" "luke" "john"
##  $ three: int [1:4, 1:5] 1 2 3 4 5 6 7 8 9 10 ...
##  $ four : logi TRUE
##  $ five :List of 3
##   ..$ : num [1:51, 1:2] 3481823 496387 4582842 2120139 26955438 ...
##   .. ..- attr(*, "dimnames")=List of 2
##   .. .. ..$ : NULL
##   .. .. ..$ : chr [1:2] "vap" "tv"
##   ..$ : num [1:51] 3481823 496387 4582842 2120139 26955438 ...
##   ..$ : num 3
\end{verbatim}

\begin{Shaded}
\begin{Highlighting}[]
\NormalTok{my.crazy.list}
\end{Highlighting}
\end{Shaded}

\begin{verbatim}
## $one
## [1] 1 2 3
## 
## $two
## [1] "matthew" "mark"    "luke"    "john"   
## 
## $three
##      [,1] [,2] [,3] [,4] [,5]
## [1,]    1    5    9   13   17
## [2,]    2    6   10   14   18
## [3,]    3    7   11   15   19
## [4,]    4    8   12   16   20
## 
## $four
## [1] TRUE
## 
## $five
## $five[[1]]
##            vap      tv
##  [1,]  3481823      NA
##  [2,]   496387  238307
##  [3,]  4582842 1553032
##  [4,]  2120139  780409
##  [5,] 26955438 8899059
##  [6,]  3617942 1586105
##  [7,]  2673154 1162391
##  [8,]   652189  258053
##  [9,]   472143  122356
## [10,] 14085749 4884544
## [11,]  6915512 2143845
## [12,]   995937  348988
## [13,]  1073799  458927
## [14,]  9600372 3586292
## [15,]  4732010 1719351
## [16,]  2265860 1071509
## [17,]  2068253  864083
## [18,]  3213141 1370062
## [19,]  3188765  954896
## [20,]  1033632      NA
## [21,]  4242214 1809237
## [22,]  4997677 2243835
## [23,]  7620982 3852008
## [24,]  3908159 2217552
## [25,]  2139918      NA
## [26,]  4426278 2178278
## [27,]   731365  411061
## [28,]  1321923  610499
## [29,]  1870315  586274
## [30,]  1012033  418550
## [31,]  6598368 2315643
## [32,]  1452962  568597
## [33,] 14838076 4703830
## [34,]  6752018 2036451
## [35,]   494923  220479
## [36,]  8697456 4184072
## [37,]  2697855      NA
## [38,]  2850525 1399650
## [39,]  9612380      NA
## [40,]   824854  392882
## [41,]  3303593 1117311
## [42,]   594599  341105
## [43,]  4636679 1868363
## [44,] 17038979      NA
## [45,]  1797941  582561
## [46,]   487900  263025
## [47,]  5841335 2398589
## [48,]  4876661 2085074
## [49,]  1421717  473014
## [50,]  4257230 2183155
## [51,]   392344  196217
## 
## $five[[2]]
##  [1]  3481823   496387  4582842  2120139 26955438  3617942  2673154
##  [8]   652189   472143 14085749  6915512   995937  1073799  9600372
## [15]  4732010  2265860  2068253  3213141  3188765  1033632  4242214
## [22]  4997677  7620982  3908159  2139918  4426278   731365  1321923
## [29]  1870315  1012033  6598368  1452962 14838076  6752018   494923
## [36]  8697456  2697855  2850525  9612380   824854  3303593   594599
## [43]  4636679 17038979  1797941   487900  5841335  4876661  1421717
## [50]  4257230   392344
## 
## $five[[3]]
## [1] 3
\end{verbatim}

But this can be quite convoluted. Instead, when we create our list, we
can give each element a name:

\begin{Shaded}
\begin{Highlighting}[]
\NormalTok{my.crazy.list <-}\StringTok{ }\KeywordTok{list}\NormalTok{(}\DataTypeTok{one=}\NormalTok{vector1,}\DataTypeTok{two=}\NormalTok{gospels, }\DataTypeTok{three=}\NormalTok{my.matrix, }\DataTypeTok{four=}\OtherTok{TRUE}\NormalTok{, }\DataTypeTok{five=}\NormalTok{list.a)}
\KeywordTok{str}\NormalTok{(my.crazy.list)}
\end{Highlighting}
\end{Shaded}

\begin{verbatim}
## List of 5
##  $ one  : num [1:3] 1 2 3
##  $ two  : chr [1:4] "matthew" "mark" "luke" "john"
##  $ three: int [1:4, 1:5] 1 2 3 4 5 6 7 8 9 10 ...
##  $ four : logi TRUE
##  $ five :List of 3
##   ..$ : num [1:51, 1:2] 3481823 496387 4582842 2120139 26955438 ...
##   .. ..- attr(*, "dimnames")=List of 2
##   .. .. ..$ : NULL
##   .. .. ..$ : chr [1:2] "vap" "tv"
##   ..$ : num [1:51] 3481823 496387 4582842 2120139 26955438 ...
##   ..$ : num 3
\end{verbatim}

\begin{Shaded}
\begin{Highlighting}[]
\KeywordTok{names}\NormalTok{(my.crazy.list)}
\end{Highlighting}
\end{Shaded}

\begin{verbatim}
## [1] "one"   "two"   "three" "four"  "five"
\end{verbatim}

Manipulating lists is similar to other manipulations in R, the new one
is using double brackets \texttt{[[]]} to access an element of a list.

\begin{Shaded}
\begin{Highlighting}[]
\CommentTok{# there are several ways to access/add to/subtract from a list}
\NormalTok{my.crazy.list[[}\DecValTok{1}\NormalTok{]]}
\end{Highlighting}
\end{Shaded}

\begin{verbatim}
## [1] 1 2 3
\end{verbatim}

\begin{Shaded}
\begin{Highlighting}[]
\NormalTok{my.crazy.list}\OperatorTok{$}\NormalTok{one}
\end{Highlighting}
\end{Shaded}

\begin{verbatim}
## [1] 1 2 3
\end{verbatim}

\begin{Shaded}
\begin{Highlighting}[]
\NormalTok{my.crazy.list[}\DecValTok{1}\NormalTok{]}
\end{Highlighting}
\end{Shaded}

\begin{verbatim}
## $one
## [1] 1 2 3
\end{verbatim}

\begin{Shaded}
\begin{Highlighting}[]
\NormalTok{my.crazy.list[}\StringTok{"one"}\NormalTok{]}
\end{Highlighting}
\end{Shaded}

\begin{verbatim}
## $one
## [1] 1 2 3
\end{verbatim}

\begin{Shaded}
\begin{Highlighting}[]
\NormalTok{my.crazy.list}\OperatorTok{$}\NormalTok{dracula <-}\StringTok{ "dracula"}
\NormalTok{my.crazy.list }\CommentTok{# now we have added another element}
\end{Highlighting}
\end{Shaded}

\begin{verbatim}
## $one
## [1] 1 2 3
## 
## $two
## [1] "matthew" "mark"    "luke"    "john"   
## 
## $three
##      [,1] [,2] [,3] [,4] [,5]
## [1,]    1    5    9   13   17
## [2,]    2    6   10   14   18
## [3,]    3    7   11   15   19
## [4,]    4    8   12   16   20
## 
## $four
## [1] TRUE
## 
## $five
## $five[[1]]
##            vap      tv
##  [1,]  3481823      NA
##  [2,]   496387  238307
##  [3,]  4582842 1553032
##  [4,]  2120139  780409
##  [5,] 26955438 8899059
##  [6,]  3617942 1586105
##  [7,]  2673154 1162391
##  [8,]   652189  258053
##  [9,]   472143  122356
## [10,] 14085749 4884544
## [11,]  6915512 2143845
## [12,]   995937  348988
## [13,]  1073799  458927
## [14,]  9600372 3586292
## [15,]  4732010 1719351
## [16,]  2265860 1071509
## [17,]  2068253  864083
## [18,]  3213141 1370062
## [19,]  3188765  954896
## [20,]  1033632      NA
## [21,]  4242214 1809237
## [22,]  4997677 2243835
## [23,]  7620982 3852008
## [24,]  3908159 2217552
## [25,]  2139918      NA
## [26,]  4426278 2178278
## [27,]   731365  411061
## [28,]  1321923  610499
## [29,]  1870315  586274
## [30,]  1012033  418550
## [31,]  6598368 2315643
## [32,]  1452962  568597
## [33,] 14838076 4703830
## [34,]  6752018 2036451
## [35,]   494923  220479
## [36,]  8697456 4184072
## [37,]  2697855      NA
## [38,]  2850525 1399650
## [39,]  9612380      NA
## [40,]   824854  392882
## [41,]  3303593 1117311
## [42,]   594599  341105
## [43,]  4636679 1868363
## [44,] 17038979      NA
## [45,]  1797941  582561
## [46,]   487900  263025
## [47,]  5841335 2398589
## [48,]  4876661 2085074
## [49,]  1421717  473014
## [50,]  4257230 2183155
## [51,]   392344  196217
## 
## $five[[2]]
##  [1]  3481823   496387  4582842  2120139 26955438  3617942  2673154
##  [8]   652189   472143 14085749  6915512   995937  1073799  9600372
## [15]  4732010  2265860  2068253  3213141  3188765  1033632  4242214
## [22]  4997677  7620982  3908159  2139918  4426278   731365  1321923
## [29]  1870315  1012033  6598368  1452962 14838076  6752018   494923
## [36]  8697456  2697855  2850525  9612380   824854  3303593   594599
## [43]  4636679 17038979  1797941   487900  5841335  4876661  1421717
## [50]  4257230   392344
## 
## $five[[3]]
## [1] 3
## 
## 
## $dracula
## [1] "dracula"
\end{verbatim}

\begin{Shaded}
\begin{Highlighting}[]
\CommentTok{# We can repeat this accessing method}
\NormalTok{my.crazy.list[[}\DecValTok{3}\NormalTok{]][}\DecValTok{1}\NormalTok{,] }\CommentTok{# first row of my.matrix}
\end{Highlighting}
\end{Shaded}

\begin{verbatim}
## [1]  1  5  9 13 17
\end{verbatim}

\begin{Shaded}
\begin{Highlighting}[]
\NormalTok{my.matrix[}\DecValTok{1}\NormalTok{,]  }\CommentTok{#the same}
\end{Highlighting}
\end{Shaded}

\begin{verbatim}
## [1]  1  5  9 13 17
\end{verbatim}

However, you cannot do math on lists directly (note that this is set to
\texttt{eval\ =\ FALSE}, since if we ran it, it throws an error and the
document doesn't compile):

\begin{Shaded}
\begin{Highlighting}[]
\NormalTok{my.crazy.list }\OperatorTok{+}\DecValTok{2} \CommentTok{# not so much}
\NormalTok{my.crazy.list[[}\DecValTok{3}\NormalTok{]] }\OperatorTok{+}\StringTok{ }\DecValTok{2}
\end{Highlighting}
\end{Shaded}


\end{document}
